We demonstrate that \method is a generic large scale machine learning system by
applying it over diverse set of real world problems that are non-trivial to
solve. We describe here a set of problems in machine learning from the sub-areas
of graphical models, natural language processing, computer vision and
computational social sciences. These applications are real world problems that
involve non-trivial complexity described in section~\ref{sec:complexQues}

\subsection{Latent dirichlet allocation (\lda)}
We descibed \lda in detail in section~\ref{sec:mdAbstract}. We further elaborate
on the problem and its application here. The assumption that there are a fixed
set of topics and each document is composed of topics with certain weights is
helpful in search engine queries and information retrieval~\cite{Wei:LDM}. The
search engine can use topics as part of the indexing strategy to keep similar
documents together. This helps in retrieving faster and accurate results for
search queries. A similar strtegy is used by libraries for efficient
storage of documents~\cite{Newman:ETM} in digital form. Besides prevalent in
text mining an natural language processing, it is one of the most common
building blocks of complex graphical models~\cite{}

\subsection{Dictionary learning (\dl)}
\subsection{Mixed membership stochastic block models (\mmsb)}

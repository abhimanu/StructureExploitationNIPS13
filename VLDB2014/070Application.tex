We demonstrate that \method is a generic large scale machine learning system by
applying it over diverse set of challenging, real world problems. We describe
here a set of problems in machine learning from the sub-areas of graphical
models, natural language processing, computer vision and computational social
sciences. These applications are real world problems that involve non-trivial
complexity described in Section~\ref{sec:complexQues}.

\subsection{Latent dirichlet allocation (\lda)}
We described \lda in detail in Section~\ref{sec:mdAbstract}. We further
elaborate on this model and its application here. The assumption that there
are a fixed set of topics and each document is composed of topics with certain
probabilistic weights is helpful in search engine queries and information
retrieval. The search engine can use topics as part of their indices to keep
similar documents together. This helps in retrieving faster and accurate results for
search queries~\cite{Wei:LDM}. A similar strtegy is used by libraries for
efficient storage of documents~\cite{Newman:ETM} in digital form. Besides prevalent in
text mining and natural language processing, it is one of the most common
building blocks of complex graphical models such as nested
chinese restaurant process~\cite{Blei:NCR} used in genetics for clustering
micro-array data~\cite{Qin:CMG}, hierarchical dirichlet process~\cite{hdp:2006} 
used for tracking trending topics~\cite{Gao:TCT}, among other things.

This model uses non-negativity, simplex as well as distributed simplex
constraints defined in Section~\ref{par:Simplex Constraints}. We will see in
Section~\ref{sec:eval} as to how these constraints affect run time
and convergence quality.

%\begin{table}[h!]
%\centering
%\scalebox{0.95}{
%\begin{tabular}{c|c|c|c|} %\hline
%Dataset  & Dimensions & Data points & Size(GB) \\ \hline
%\nytimes  & $0.3*10^6\times$102,660 & $0.1*10^9$ &  1.49  \\ \hline
%\snytimes{4} & $1,2*10^6\times$102,660 & $0.4*10^9$ &  6.08  \\ \hline
%\snytimes{16} & $4.8*10^6\times$102,660 & $1.6*10^9$ &  25.12  \\ \hline
%\snytimes{32} & $9.6*10^6\times$102,660 & $3.2*10^9$ &  50.5  \\ \hline
%\snytimes{64} & $19.2*10^6\times$102,660 & $6.4*10^9$ &  103.4  \\ \hline
%\snytimes{256} & $76.8*10^6\times$102,660 & $25.6*10^9$ &  421.42  \\ \hline
%\hline
%\imagenet & $1.26*10^6\times$1,000 & $1.25*10^9$ & 16.4  \\ \hline
%\simagenet{0.5} & $0.63*10^6\times$1,000 & $0.63*10^9$ & 8.0  \\ \hline
%\simagenet{1.5} & $1.89*10^6\times$1,000 & $1.89*10^9$ & 25.1  \\ \hline
%\simagenet{2} & $2.51*10^6\times$1,000 & $2.51*10^9$ & 33.8  \\ \hline
%\simagenet{2.5} & $3.15*10^6\times$1,000 & $3.15*10^9$ & 42.6  \\ \hline
%\simagenet{3} & $3.77*10^6\times$1,000 & $3.77*10^9$ & 51.3  \\ \hline
%\simagenet{3.5} & $4.40*10^6\times$1,000 & $4.40*10^9$ & 60.0  \\ \hline
%\simagenet{4} & $5.03*10^6\times$1,000 & $5.03*10^9$ & 68.8  \\ \hline
%\simagenet{8} & $10.03*10^6\times$1,000 & $10.03*10^9$ & 138.8  \\ \hline
%\simagenet{16} & $20.13*10^6\times$1,000 & $20.13*10^9$ & 288.7  \\ \hline
%\hline
%\twitter & 281,903$\times$ 281,903 & $2.3*10^6$ & 0.03 \\ \hline
%\swebgraph{1} & 281,903$\times$ 281,903 & $0.39*10^9$ & 6.0 \\ \hline
%\swebgraph{2} & 281,903$\times$ 281,903 & $0.79*10^9$ & 12.1 \\ \hline
%\swebgraph{3} & 281,903$\times$ 281,903 & $1.19*10^9$ & 18.1 \\ \hline
%\swebgraph{4} & 281,903$\times$ 281,903 & $1.59*10^9$ & 24.2 \\ \hline
%\swebgraph{5} & 281,903$\times$ 281,903 & $1.98*10^9$ & 30.2 \\ \hline
%\swebgraph{6} & 281,903$\times$ 281,903 & $2.38*10^9$ & 36.3 \\ \hline
%\swebgraph{7} & 281,903$\times$ 281,903 & $2.78*10^9$ & 42.3 \\ \hline
%\swebgraph{8} & 281,903$\times$ 281,903 & $3.17*10^9$ & 48.3 \\ \hline
%\swebgraph{16} & 281,903$\times$ 281,903 & $6.35*10^9$ & 96.7 \\ \hline
%\hline
%\end{tabular}
%}
%\caption{Dimension, size and data points statistics for different datasets. 
%The data point is an edge in case of \mmsb, an entry in the pixel matrix 
%for \dl and a word count in case of \lda input matrix.	
%The exact figures are rounded off for simplicity. Size is the file size in
%gigabytes. The biggest dataset (\snytimes{256}) is of size approximately 0.5
%terabytes.}
%\label{tab:dataset}
%\end{table}

\subsection{Sparse Dictionary learning (\sdl)}
Dictionary Learning (\dl) is a classical model in computer vision used for image
denoising, restoration~\cite{Mairal07sparserepresentation} and
classification~\cite{RamirezSS10}. Given a signal matrix $Y_{m,n}$ where each
column of $Y$ is an observation of a signal, with $n$ such observation. We
would like represent each observed signal (column of $Y$) $Y_j$ as weighted
combination of $K$ basis vectors $D_k$ (or columns). The basis vectors $D_k$s
form the columns of the matrix $D$ called dictionary. The underlying assumption
is that there is an inherent set of basis vectors that is the building block
of any signal observed. This helps in image classification as the incoming
unlabeled image query can be matched with labled image's weights of basis vector
to predict a label. Mathematically we are solving the following query:

% {\small
\begin{align}
&\arg\min_{\alpha,D} L(Y,\alpha,D) =\frac{1}{2}||Y-D\alpha||_2^2 + \lambda||\alpha||_1
\label{eqn:dictL}\\
&\text{s.t.} \; \forall j, D_j^TD_j \leq 1 \nonumber
\end{align}
% }
$\alpha_j$, the $j$th column of $\alpha$, is the weight coefficient for basis
vectors in signal $j$. The $\lambda||\alpha||_1$ in equation~\ref{eqn:dictL} as
discussed in Section~\ref{par:Sparse Models}, helps in removing the noise from the
signal~\cite{Mairal07sparserepresentation}.
 
\subsection{Mixed network decomposition models (\mmsb)}
\mmsb models or multi-role models are very useful in social sciences. The premise
is that in a social network people are part of many communities simultaneously. 
This runs counter to traditional clustering techniques (spectral clustering, k-means) 
that assume non-overlapping clusters. This formulation is important in various social
settings where people play different roles in different communities. This model can 
find the affinity of a given person (probability that he is a member) to 
pre-defined set of communities. Given an $N\times N$ adjacency matrix of $N$ 
users and $K$ communities, decomposing it into matrices $\Theta_{N,K}$ 
and $B_{K,K}$ gives membership probability ($\theta_{u,*}$) of each user $u$ 
for $K$ communities. More concretely:
\begin{align}
&\arg\min_{\Theta, B} L(Y,\Theta,B) = ||Y-\Theta B\Theta^T||_2 \label{eqn:mmsb}\\
&\text{s.t.} \; \forall u \in \{1\ldots N\}, 
\theta_u^T\theta_u = 1, \& \theta_{u,k}\geq 0 \forall k\in\{1\ldots K\} \nonumber
\end{align}

These probabilities can be used to infer like-dislikes, political and social 
affinities etc. and various other social traits of a person. 
It also very helpful in discovering social clusters  that can go under the radar of
traditional clustering approaches~\cite{Airoldi:2008}. 
%\alex{abhi: fill this in}

\ourmethod provides a generic partitioning and schedulng strategy
that is distributed on data as well as model size. Its firm theoretical
standing ensures better annswers. 
And its distributivity over data and model guarantees
faster answers. These attributes have been justified emprically ( 
section ~\ref{sec:eval}) as well as theoretically~\cite{theoryResult}.
Though there are certain areas that need further  
improvement. 

\ourmethod does not scale well in number of machines if the data 
size is small. In such a scenario the synchonization time dominates every 
thing else (figure~\ref{fig:piechart}). This stems from the 
limitation of using hadoop . A system that can ensure few inter-machine 
communications (e.g. SSP) can help \ourmethod overcome this. We plan to 
explore this as future work.  Another problem is: in case of pathalogically skewed load distribution 
between the machines or in case a mahine dies altogether, the guarantess 
of \ourmethod may not hold. These can be solved by sorting the data for 
fair load scheduling prior to 
running \ourmethod. 
